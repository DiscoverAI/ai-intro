\newacronym{ai}{AI}{Artificial Intelligence}

\newacronym{gofai}{GOFAI}{Good, Old Fashioned Artificial Intelligence}

\newacronym{csp}{CSP}{Constraint Satisfaction Problem}

\newacronym{strips}{STRIPS}{Stanford Research Institute Problem Solver}

\newacronym{dcg}{DCG}{Definite Clause Grammar}

\newacronym{gan}{GANs}{Generial Adversial Networks}

\newacronym{dqn}{DQN}{Deep Q-network}

\newacronym{neat}{NEAT}{NeuroEvolution of Augmenting Topologies}

\newglossaryentry{symbolic-ai}
{
  name={Symbolic AI},
  description={is a AI paradigm solving complex problems the way humans would. Modeling the problem space with symbols and use computations to transform them into a solution}
}

\newglossaryentry{expert-system}
{
  name={Expert Systems},
  description={Synonym of symbolic AI},
  see={symbolic-ai}
}

\newglossaryentry{knowledge-based-ai}
{
  name={Knowledge based AI},
  description={Synonym of symbolic AI},
  see={symbolic-ai}
}

\newglossaryentry{knowledge-graph}
{
  name={Knowledge graph},
  description={Synonym of symbolic AI},
  see={symbolic-ai}
}

\newglossaryentry{rules-engine}
{
  name={Rules engine},
  description={Synonym of symbolic AI},
  see={symbolic-ai}
}

\newglossaryentry{prolog}
{
  name={PROLOG},
  description={Logical pogramming language. For more information consider: \url{http://www.learnprolognow.org/}}
}

\newglossaryentry{sub-symbolic-ai}
{
  name={Sub-symbolic AI},
  description={is a AI paradigm solving complex problems in a more low level system where symbols represent abstract structures that enable cognitive functions}
}

\newglossaryentry{connectionism}
{
  name={Connectionism},
  description={is using graph theory to mimic neurobiological features. Each node in the graph is called neuron. Neurons are connected in specific ways identifying the overall type of network.}
}