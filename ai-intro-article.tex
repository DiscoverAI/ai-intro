\documentclass[jou,apacite]{apa6}

\title{A short introduction to artificial intelligence}
\shorttitle{AI intro}

\author{Daniel Schruhl}
\affiliation{ThoughtWorks}

\abstract{A short introduction into the general topic of artificial intelligence. This should help address some aspects of artificial intelligence and define them on a shallow level to give a starting point into this topic.}

\begin{document}
\maketitle    
                        
\section{Motivation}
There are a lot of definitions for artificial intelligence (AI). Most of them are based on a definition of intelligence which in itself is already difficult to define. A simple approach would be to say that humans solving complex problems (with thoughts) is considered intelligent. So making programs or machines pursue the way humans might solve problems can be considered as artificial intelligence.

Another aspect of AI is often that programs associated with it are considered intelligent in itself because it is hard to understand how they work. This is mainly due to the fact that they work in ways that do not come to you straight away. That is often also considered as intelligent.
But in programming it and actually coming up with the idea of a program that is considered to be an AI, the myth of this intelligence is already debunked and does not pose any intelligence anymore. This illustrates the possible paradoxon in AI.

AI can be used for image recognition, for controllig machines, to play games or to detect fraud. It can be used in nearly any domain and has already found great usage in medicine or commercial scenarios. Using AI can give a company a major advantage against competitors. Companies like Netflix have embraced AI \cite{Gomez-Uribe2015} and have established a market leading position also backed by AI.

\section{Types of artificial intelligence}
 - symbolic
  - rule based
  - logics
  - SHRDLU
  - search algorithms, constraints
  - expert systems
 - sub symbolic
  - supervised
  - unsupervised
  - semi supervised
   - neural nets, deep learning
  - reinforcement learning
  - evolutionary algorithms

\section{Current state of AI and further topics}
 - neural networks, capsule networks, autoencoders, attention, gans, ...
 - neuroevolution

\section{Where to follow up}
deeplearning book
further readings

\bibliography{ai-intro-article}

\end{document}
